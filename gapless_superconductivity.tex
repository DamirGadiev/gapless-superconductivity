\documentclass{article}
%\documentclass[journal]{IEEEtran}
%\documentclass{report}
%\documentclass{ActaOulu}

\usepackage{graphicx}
\usepackage{mathtools}

\begin{document}

\title{Gapless Superconductivity Overview}

\maketitle

\begin{abstract}
This document contains the brief overview of the gapless superconductivity results
shown in Maki's review.
\end{abstract}


%\chapter{First Chapter}

\section{Introduction}

Superconducting energy gap - it's energy range of supressed density of the electrons
states around Fermi energy, this feature usually considered as key attribute of 
the superconductivity. Abrikosov and Gor'kov had discovered the "gapless behavior"
during the study of the effect of magnetic impurities on superconductivity.

Common aspects:
\begin{enumerate}
\item The external perturbation breaks the time-reversal symmetry of the electron system.
\item Dissipation mechanism, mixing the time-reversed states is involved.
\end{enumerate}

\section{Time-reversal symmetry}

Superconductivity is understood as the corellation of electrons in pairs formed with
mutually time-reversed states(i.e., opposite momenta and spins). Term "time-reversal
symmetry", used here, referring to the symmetry of the electron system.

\subsection {4D-Repersentation}

Reduced Hamiltonian:
\begin{equation}
   H=\sum_{\substack{
    k,\sigma
   }}
   \epsilon_k
   n_{k\sigma}
   +
   \sum_{\substack{
      k,l
    }}
    V_{k,l}
    c^{\dagger}_{k, \uparrow}
    c^{\dagger}_{-k, \downarrow}
    c_{-l, \uparrow}
    c_{l, \downarrow}
\end{equation}

Assuming that states for the fermions are given:
\begin{equation}
    |\uparrow\rangle = 
    \begin{pmatrix}
        1 \\
        0 \\
    \end{pmatrix}
    \textrm{, and }
    |\downarrow\rangle = 
    \begin{pmatrix}
        0 \\
        1 \\
    \end{pmatrix}
\end{equation}

Than operator of quantity of the particles in state s should be given:
\begin{equation}
    n_s = n_{k\sigma} = c^{\dagger}_{k,\sigma}c_{k,\sigma} =
    \begin{pmatrix}
        0 & 0 \\
        0 & 1 \\
    \end{pmatrix}
\end{equation}

Creation, annihilation operators are:
\begin{equation}
    c^{\dagger}_{k, \sigma} = \begin{pmatrix}
        0 & 0 \\
        1 & 0 \\
    \end{pmatrix}
    \textrm{, and }
    c_{k, \sigma} = \begin{pmatrix}
        1 & 0 \\
        0 & 0 \\
    \end{pmatrix}
\end{equation}


Field operators of these states should be given as the Fourier-transform.
\begin{equation}
    \psi^{\dagger}(x) =
    \int{
        \frac{d^3k}{(2\pi)^\frac{1}{2}}e^{-ikx}c^{\dagger}_k
    }
    \textrm{, and }
    \psi(x) =
    \int{
        \frac{d^3k}{(2\pi)^\frac{1}{2}}e^{ikx}c_k
    }
\end{equation}


Let's use a spinor representation of the single-particle (or hole) state as follows:
\begin{equation}
\Psi(x)=
\begin{pmatrix}
    \psi_\uparrow(x) \\
    \psi_\downarrow(x) \\
    \psi^\dagger_\uparrow(x) \\
    \psi^\dagger_\downarrow(x) \\
\end{pmatrix}
\textrm{, and }
\Psi^\dagger(x)=
\begin{pmatrix}
    \psi^\dagger_\uparrow(x) & \psi^\dagger_\downarrow(x) & \psi_\uparrow(x) & \psi_\downarrow(x)
\end{pmatrix}
\end{equation}

Hamiltonian in terms of the field operators.
\begin{equation}
    \mathcal{H}_0=
    \sum_{\substack{
        \sigma=\uparrow,\downarrow
    }}
    \int{
        \psi^\dagger_\sigma\left(-\frac{1}{2m}\nabla^2-\mu\right)\psi_\sigma{d^3}x
    }
    {\textrm{+V}}
    \int{
        (\psi^\dagger(x)\psi(x))(\psi^\dagger(x)\psi(x)){d^3}x
    }
\end{equation}

\end{document}
